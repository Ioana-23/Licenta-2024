\chapter{Introduction}
\label{intro}

\section{Motives for studying breast cancer detection}


Breast cancer is known to be the most common cancer among women~\cite{link1}. The American Cancer Society’s estimates for breast cancer in the United States alone for 2023 are that about 297,790 new cases of invasive breast cancer will be diagnosed in women and about 43,700 women will die from breast cancer~\cite{link2}. Since 1989, breast cancer death rates have been decreasing, believed to be a consequence of early detection and increased awareness, as well as better treatments. This progress seems to have slightly stopped~\cite{link2}.

The main objective of this project is to understand and implement ways to detect the presence of cancerous, benign, or precancerous tumors in the breast in an efficient way using machine learning. This would minimize the time doctors spend studying thousands of breast screenings to label them accordingly and aid early detection.

\section{Paper structure}

The definitions of the two problems, classification and detection models, are represented in chapter \ref{chap:ch2}. Along with the problem definition, the documentation of other papers and datasets focused on detecting breast cancer is also present in chapter \ref{chap:ch2}. Chapter \ref{chap:ch3} contains the details of the two architectures used for experimenting with, GoogLeNet and EfficientDet, by showcasing original concepts and layers used in the structure of the models. The experiments made are documented in chapter \ref{chap:ch4}, along with a presentation of the dataset used and data processing. The design and implementation of the intelligent system are written in chapter \ref{chap:ch5}, and the conclusion of the experiments and possible improvements for achieving better performance are presented in chapter \ref{chap:ch6}.

\section{Original contributions}

The following thesis provides original contributions to the world of machine learning. Some of these contributions are: comparative analysis of the literature by comparing datasets, models and performances and training and testing of various classification and detection models, followed by a bi-objective comparison. One criterion is the loss function; another possible criterion is the input type (2D or 3D). Along with these, the design and implementation of the system are another contribution because of the intelligent system that integrates the best classification and detection models. Also, the comparative analysis of the trained models with SOTA models represents another original contribution of this thesis.

\section{Use of generative AI instruments}

During the process of writing this thesis, I used a grammar checker, QuillBot, to flag the grammatical errors made in writing this thesis. Each modification proposed by the tool was approved by me.