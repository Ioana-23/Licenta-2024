\chapter{Introduction}
\label{intro}

\section{Motives for studying breast cancer detection}


Breast cancer is known to be the most common cancer among women~\cite{link1}. The American Cancer Society’s estimates for breast cancer in the United States alone for 2023 are that about 297,790 new cases of invasive breast cancer will be diagnosed in women and about 43,700 women will die from breast cancer~\cite{link2}. Since 1989, breast cancer death rates have been decreasing, believed to be a consequence of early detection and increased awareness, as well as better treatments. This progress seems to have slightly stopped~\cite{link2}.

The main objective of this project is to understand and implement ways to detect the presence of cancerous, benign, or precancerous tumors in the breast in an efficient way using machine learning. This would minimize the time doctors spend studying thousands of breast screenings to label them accordingly and aid early detection.

\section{Paper strcuture and original contributions}

\textcolor{green}{TODOs: Add the contributions of the paper and the structure of the paper. For example, the contributions could be:\\
1. comparative analysis of the literature (datasets, models, performances)\\
2. training \& testing of various classification/detection models, followed by a bi-objective comparison (one criterion is the recognition performance, another criterion is the loss function; another possible criterion is the input type (2D or 3D))\\
3. comparative analysis of the trained models with SOTA models\\
4. design and implementation of the system = the intelligent system that integrates the best recognition \& detection models\\
}

\textcolor{green}{TODOs: Add the structure of the paper.}